% Preamble file - general definitions, package loading, etc.

%=================================
% Load packages
\usepackage{amssymb,amsmath}
\usepackage{graphicx}
\usepackage{url}
\usepackage{tikz}
\usetikzlibrary{mindmap,trees,arrows}
\usepackage{fancyvrb}
\usepackage[english]{babel}
\usepackage[latin1]{inputenc}
\usepackage{subfigure}
\usepackage{times}
\usepackage[T1]{fontenc}
\usepackage{cancel}
\usepackage{color}
\usepackage{listings}


%=================================
% Listings package options for R syntax highlighting
 \lstset{ 
  language=R,                    		 	% the language of the code
  basicstyle=\tiny\ttfamily, 			% the size of the fonts that are used for the code
  numbers=left,                   			% where to put the line-numbers
%  numberstyle=\tiny\color{blue},	% the style that is used for the line-numbers
stepnumber=1,                   		% the step between two line-numbers. If it's 1, each line
                                  					% will be numbered
  numbersep=5pt,                  		% how far the line-numbers are from the code
  backgroundcolor=\color{white},  % choose the background color. You must add \usepackage{color}
  showspaces=false,               			% show spaces adding particular underscores
  showstringspaces=false,         		% underline spaces within strings
  showtabs=false,                 			% show tabs within strings adding particular underscores
  frame=single,                   			% adds a frame around the code
  rulecolor=\color{black},        		% if not set, the frame-color may be changed on line-breaks within not-black text (e.g. commens (green here))
  tabsize=2,                      				% sets default tabsize to 2 spaces
  captionpos=b,                   			% sets the caption-position to bottom
  breaklines=true,               			 % sets automatic line breaking
  breakatwhitespace=false,        			% sets if automatic breaks should only happen at whitespace
%  keywordstyle=\color{RoyalBlue},      % keyword style
%  commentstyle=\color{YellowGreen},   % comment style
%  stringstyle=\color{ForestGreen}      % string literal style
} 

%=================================
% Set mode
\mode<presentation>
{
	\usetheme{Madrid}
	\usecolortheme{whale}
	\useoutertheme{infolines}
	\setbeamercovered{invisible}
}

% Get rid of nav bar
\beamertemplatenavigationsymbolsempty

% Insert frame number at bottom of the page.
\usefoottemplate{\hfil\tiny{\color{black!90}\insertframenumber}} 

%=================================
% Define new commands

\newcommand\Real{{\mathbb{R}}}
%\newcommand{\vi}{\vspace{0.6\baselineskip}}
%\newcommand{\goodgap}{\hspace{\subfigtopskip}\hspace{\subfigbottomskip}}


% Equation environments
\newcommand{\beq}{\begin{equation}}
\newcommand{\eq}{\end{equation}}
\newcommand{\beqs}{\begin{equation*}}
\newcommand{\eqs}{\end{equation*}}
\newcommand{\beqn}{\begin{eqnarray}}
\newcommand{\eqn}{\end{eqnarray}}

% Bold variables
\newcommand{\mbf}[1]{\ensuremath{\mathbf{#1}}}

% Itemization
\newcommand{\bitem}{\begin{itemize}}
\newcommand{\eitem}{\end{itemize}}
\newcommand{\spitem}{\vskip 1em\item}
\newcommand{\bitems}{\begin{itemize}\item}
\newcommand{\benums}{\begin{enumerate}\item}
\newcommand{\eenum}{\end{enumerate}}

% color blocks
\newenvironment{colorblock}[2]{%
\setbeamercolor{block title}{#2}
\begin{block}{#1}}{\end{block}}

% Vertical spacing
\newcommand{\vone}{\vskip 1em}
\newcommand{\vhalf}{\vskip .5em}

% Frame environments
\newenvironment{ftst}[3][t]{%
\begin{frame}{environment=ftst,#1}
\frametitle{#2}
\framesubtitle{#3}}{\end{frame}}

\newenvironment{ftstf}[2]{
\begin{frame}[fragile,environment=ftstf]
\frametitle{#1}
\framesubtitle{#2}}{\end{frame}}

% colors
\definecolor{MyGray}{rgb}{0.5,0.5,0.5}
\definecolor{MyDBGray}{rgb}{0.1,0.1,0.4}
\def\defn#1{{\color{red} #1}}

% Footnote
\renewcommand{\thefootnote}{\alph{footnote}}

% Verbatim environment - using FANCYVRB package
\DefineVerbatimEnvironment%
{rcode}{Verbatim}
{fontsize=\scriptsize}

% Verbatim environment - using LISTINGS package
%\newenvironment{rcode}[#1]{
%\begin{lstlisting}[language=R, ]}{\end{lstlings}}


% Relaxed footnotes
\newcommand{\lfr}[1]{\let\thefootnote\relax\footnote{\tiny #1}}