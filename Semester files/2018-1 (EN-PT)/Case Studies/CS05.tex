\documentclass[11pt,twoside,printwatermark=false]{pinp}

%% Some pieces required from the pandoc template
\providecommand{\tightlist}{%
  \setlength{\itemsep}{0pt}\setlength{\parskip}{0pt}}

% Use the lineno option to display guide line numbers if required.
% Note that the use of elements such as single-column equations
% may affect the guide line number alignment.

\usepackage[T1]{fontenc}
\usepackage[utf8]{inputenc}

\definecolor{pinpblue}{HTML}{185FAF}  % imagecolorpicker on blue for new R logo
\definecolor{pnasbluetext}{RGB}{101,0,0} %



\title{EEE933 - Design and Analysis of Experiments}

\author[]{Case Study 05}


\setcounter{secnumdepth}{0}

% Please give the surname of the lead author for the running footer
\leadauthor{}

% Keywords are not mandatory, but authors are strongly encouraged to provide them. If provided, please include two to five keywords, separated by the pipe symbol, e.g:
 

\begin{abstract}
Experiment: Comparação de desempenho de quatro configurações de um
algoritmo de otimização, parte II.
\end{abstract}

\dates{This version was compiled on \today}


\begin{document}

% Optional adjustment to line up main text (after abstract) of first page with line numbers, when using both lineno and twocolumn options.
% You should only change this length when you've finalised the article contents.
\verticaladjustment{-2pt}

\maketitle
\thispagestyle{firststyle}
\ifthenelse{\boolean{shortarticle}}{\ifthenelse{\boolean{singlecolumn}}{\abscontentformatted}{\abscontent}}{}

% If your first paragraph (i.e. with the \dropcap) contains a list environment (quote, quotation, theorem, definition, enumerate, itemize...), the line after the list may have some extra indentation. If this is the case, add \parshape=0 to the end of the list environment.


\subsection{Apresentação}\label{apresentacao}

Algoritmos baseados em populações são uma alternativa comum para a
solução de problemas de otimização em engenharia. Tais algoritmos
normalmente consistem de um ciclo iterativo, no qual um conjunto de
soluções-candidatas ao problema são repetidamente sujeitas a operadores
de variação e seleção, de forma a promover uma exploração do espaço de
variáveis do problema em busca de um ponto de ótimo (máximo ou mínimo)
de uma dada função-objetivo.

Dentre estes algoritmos, um método que tem sido bastante utilizado nos
últimos anos é conhecido como \emph{evolução diferencial} (DE, do inglês
\emph{differential evolution})\citep{Storn1997}. De forma simplificada,
este método é composto pelos seguintes passos:

\begin{enumerate}
\def\labelenumi{\arabic{enumi}.}
\setcounter{enumi}{-1}
\tightlist
\item
  Entrada: \(N,\ n_{iter},\ recpars,\ mutpars\)
\item
  \(t \leftarrow 0\)
\item
  \(X_t \leftarrow \{\vec{x}_1, \vec{x}_2, \ldots, \vec{x}_N\}\)
  (população inicial)
\item
  \(\vec{f}_t \leftarrow f(X_t)\)
\item
  Enquanto (\(t < n_{iter}\))

  \begin{enumerate}
  \def\labelenumii{\arabic{enumii}.}
  \tightlist
  \item
    \(V_t \leftarrow\) mutação(\(X_t,\ mutpars\))
  \item
    \(U_t \leftarrow\) recombinação(\(X_t,\ V_t,\ recpars\))
  \item
    \(\vec{j}_t \leftarrow f(U_t)\)
  \item
    \((X_{t+1},\ \vec{f}_{t+1}) \leftarrow\)
    seleção(\(X_t,\ U_t,\ \vec{f}_t,\ \vec{j}_t\))
  \item
    \(t \leftarrow t + 1\)
  \end{enumerate}
\item
  Saída: \((X_t,\ \vec{f}_t)\)
\end{enumerate}

Suponha que um pesquisador está interessado em investigar o efeito de
quatro configurações distintas deste algoritmo em seu desempenho para
uma dada classe de problemas de otimização.

\subsection{Atividades}\label{atividades}

Como forma de análise deste problema, cada equipe terá como tarefa a
comparação experimental de quatro configurações em uma classe de
problemas, representada por um conjunto de instâncias de teste. O
objetivo deste estudo é responder às seguintes perguntas:

\begin{center}\textit{Há alguma diferença no desempenho médio do algoritmo quando equipado com estas diferentes configurações, \textbf{para a classe de problemas de interesse}? Caso haja, qual a melhor configuração (ou melhores configurações) em termos de desempenho médio (atenção: quanto \textit{menor} o valor retornado, melhor o algoritmo), e qual a magnitude das diferenças encontradas? Há alguma configuração que deva ser recomendada em relação às outras?}\end{center}

Os seguintes parâmetros experimentais são dados para este estudo:

\begin{itemize}
\tightlist
\item
  Mínima diferença de importância prática (padronizada):
  (\(d^* = \delta^*/\sigma\)) = 0.5
\item
  Significância desejada: \(\alpha = 0.05\)
\item
  Potência mínima desejada (para o caso \(d = d^*\)):
  \(\pi = 1 - \beta = 0.8\) \newpage
\end{itemize}

\subsection{Informações operacionais}\label{informacoes-operacionais}

Para a execução dos experimentos, instale os pacotes \texttt{ExpDE}
\citep{Campelo2016} e \texttt{smoof} \citep{smoof}:

\begin{Shaded}
\begin{Highlighting}[]
\KeywordTok{install.packages}\NormalTok{(}\StringTok{"ExpDE"}\NormalTok{)}
\KeywordTok{install.packages}\NormalTok{(}\StringTok{"smoof"}\NormalTok{)}
\end{Highlighting}
\end{Shaded}

A classe de funções de interesse para este teste é composta por funções
Rosenbrock \citep{rosenbrock} \citep{Fobjs} de dimensão entre 2 e 150.
Para gerar uma função de Rosenbrock de uma dada dimensão \(dim\), faça:

\begin{Shaded}
\begin{Highlighting}[]
\KeywordTok{suppressPackageStartupMessages}\NormalTok{(}\KeywordTok{library}\NormalTok{(smoof))}

\CommentTok{# FOR INSTANCE: dim = 10}
\NormalTok{dim <-}\StringTok{ }\DecValTok{10}

\NormalTok{fn <-}\StringTok{ }\ControlFlowTok{function}\NormalTok{(X)\{}
  \ControlFlowTok{if}\NormalTok{(}\OperatorTok{!}\KeywordTok{is.matrix}\NormalTok{(X)) X <-}\StringTok{ }\KeywordTok{matrix}\NormalTok{(X, }\DataTypeTok{nrow =} \DecValTok{1}\NormalTok{) }\CommentTok{# <- if a single vector is passed as X}

\NormalTok{  Y <-}\StringTok{ }\KeywordTok{apply}\NormalTok{(X, }\DataTypeTok{MARGIN =} \DecValTok{1}\NormalTok{, }
             \DataTypeTok{FUN =}\NormalTok{ smoof}\OperatorTok{::}\KeywordTok{makeRosenbrockFunction}\NormalTok{(}\DataTypeTok{dimensions =}\NormalTok{ dim))}
  
  \KeywordTok{return}\NormalTok{(Y)}
\NormalTok{\}}

\CommentTok{# testing the function on a matrix composed of 2 points}
\NormalTok{X <-}\StringTok{ }\KeywordTok{matrix}\NormalTok{(}\KeywordTok{runif}\NormalTok{(}\DecValTok{20}\NormalTok{), }\DataTypeTok{nrow =} \DecValTok{2}\NormalTok{)}
\KeywordTok{fn}\NormalTok{(X)}
\end{Highlighting}
\end{Shaded}

\begin{ShadedResult}
\begin{verbatim}
#  [1] 195.8478 156.4954
\end{verbatim}
\end{ShadedResult}

Os limites das variáveis para uma dada função Rosenbrock de dimensão
\(dim\) são dados por: \(-5\leq x_i\leq 10,~i=1,\dotsc,dim\). Para o
problema de dimensão \(dim\), os seguintes parâmetros são dados:

\begin{Shaded}
\begin{Highlighting}[]
\CommentTok{# FOR INSTANCE: dim = 10}
\NormalTok{dim <-}\StringTok{ }\DecValTok{10}

\NormalTok{selpars  <-}\StringTok{ }\KeywordTok{list}\NormalTok{(}\DataTypeTok{name  =} \StringTok{"selection_standard"}\NormalTok{)}
\NormalTok{stopcritN <-}\StringTok{ }\KeywordTok{list}\NormalTok{(}\DataTypeTok{names =} \StringTok{"stop_maxeval"}\NormalTok{, }\DataTypeTok{maxevals =} \DecValTok{5000} \OperatorTok{*}\StringTok{ }\NormalTok{dim, }\DataTypeTok{maxiter =} \DecValTok{100} \OperatorTok{*}\StringTok{ }\NormalTok{dim)}
\NormalTok{probparsN <-}\StringTok{ }\KeywordTok{list}\NormalTok{(}\DataTypeTok{name  =} \StringTok{"fn"}\NormalTok{, }\DataTypeTok{xmin  =} \KeywordTok{rep}\NormalTok{(}\OperatorTok{-}\DecValTok{5}\NormalTok{, dim), }\DataTypeTok{xmax  =} \KeywordTok{rep}\NormalTok{(}\DecValTok{10}\NormalTok{, dim))}
\NormalTok{popsizeN =}\StringTok{ }\DecValTok{5} \OperatorTok{*}\StringTok{ }\NormalTok{dim}
\end{Highlighting}
\end{Shaded}

\newpage

As configurações que deverão ser comparadas por cada equipe são dadas
por:

\begin{Shaded}
\begin{Highlighting}[]
\CommentTok{# Equipe A}

\NormalTok{## Config 1}
\NormalTok{recpars1 <-}\StringTok{ }\KeywordTok{list}\NormalTok{(}\DataTypeTok{name =} \StringTok{"recombination_arith"}\NormalTok{) }
\NormalTok{mutpars1 <-}\StringTok{ }\KeywordTok{list}\NormalTok{(}\DataTypeTok{name =} \StringTok{"mutation_rand"}\NormalTok{, }\DataTypeTok{f =} \DecValTok{4}\NormalTok{)}

\NormalTok{## Config 2}
\NormalTok{recpars2 <-}\StringTok{ }\KeywordTok{list}\NormalTok{(}\DataTypeTok{name =} \StringTok{"recombination_bin"}\NormalTok{, }\DataTypeTok{cr =} \FloatTok{0.7}\NormalTok{) }
\NormalTok{mutpars2 <-}\StringTok{ }\KeywordTok{list}\NormalTok{(}\DataTypeTok{name =} \StringTok{"mutation_best"}\NormalTok{, }\DataTypeTok{f =} \DecValTok{3}\NormalTok{)}

\NormalTok{## Config 3}
\NormalTok{recpars3 <-}\StringTok{ }\KeywordTok{list}\NormalTok{(}\DataTypeTok{name =} \StringTok{"recombination_eigen"}\NormalTok{, }
                 \DataTypeTok{othername =} \StringTok{"recombination_bin"}\NormalTok{, }\DataTypeTok{cr =} \FloatTok{0.9}\NormalTok{) }
\NormalTok{mutpars3 <-}\StringTok{ }\KeywordTok{list}\NormalTok{(}\DataTypeTok{name =} \StringTok{"mutation_best"}\NormalTok{, }\DataTypeTok{f =} \FloatTok{2.8}\NormalTok{)}

\NormalTok{## Config 4}
\NormalTok{recpars4 <-}\StringTok{ }\KeywordTok{list}\NormalTok{(}\DataTypeTok{name =} \StringTok{"recombination_sbx"}\NormalTok{, }\DataTypeTok{eta =} \DecValTok{90}\NormalTok{)}
\NormalTok{mutpars4 <-}\StringTok{ }\KeywordTok{list}\NormalTok{(}\DataTypeTok{name =} \StringTok{"mutation_best"}\NormalTok{, }\DataTypeTok{f =} \FloatTok{4.5}\NormalTok{)}
\end{Highlighting}
\end{Shaded}

\begin{Shaded}
\begin{Highlighting}[]
\CommentTok{# Equipe B}

\NormalTok{## Config 1}
\NormalTok{recpars1 <-}\StringTok{ }\KeywordTok{list}\NormalTok{(}\DataTypeTok{name =} \StringTok{"recombination_exp"}\NormalTok{, }\DataTypeTok{cr =} \FloatTok{0.6}\NormalTok{)}
\NormalTok{mutpars1 <-}\StringTok{ }\KeywordTok{list}\NormalTok{(}\DataTypeTok{name =} \StringTok{"mutation_best"}\NormalTok{, }\DataTypeTok{f =} \DecValTok{2}\NormalTok{)}
    
\NormalTok{## Config 2}
\NormalTok{recpars2 <-}\StringTok{ }\KeywordTok{list}\NormalTok{(}\DataTypeTok{name =} \StringTok{"recombination_geo"}\NormalTok{, }\DataTypeTok{alpha =} \FloatTok{0.6}\NormalTok{)}
\NormalTok{mutpars2 <-}\StringTok{ }\KeywordTok{list}\NormalTok{(}\DataTypeTok{name =} \StringTok{"mutation_rand"}\NormalTok{, }\DataTypeTok{f =} \FloatTok{1.2}\NormalTok{)}

\NormalTok{## Config 3}
\NormalTok{recpars3 <-}\StringTok{ }\KeywordTok{list}\NormalTok{(}\DataTypeTok{name =} \StringTok{"recombination_blxAlphaBeta"}\NormalTok{, }\DataTypeTok{alpha =} \FloatTok{0.4}\NormalTok{, }\DataTypeTok{beta =} \FloatTok{0.4}\NormalTok{) }
\NormalTok{mutpars3 <-}\StringTok{ }\KeywordTok{list}\NormalTok{(}\DataTypeTok{name =} \StringTok{"mutation_rand"}\NormalTok{, }\DataTypeTok{f =} \DecValTok{4}\NormalTok{)}

\NormalTok{## Config 4}
\NormalTok{recpars4 <-}\StringTok{ }\KeywordTok{list}\NormalTok{(}\DataTypeTok{name =} \StringTok{"recombination_wright"}\NormalTok{)}
\NormalTok{mutpars4 <-}\StringTok{ }\KeywordTok{list}\NormalTok{(}\DataTypeTok{name =} \StringTok{"mutation_best"}\NormalTok{, }\DataTypeTok{f =} \FloatTok{4.8}\NormalTok{)}
\end{Highlighting}
\end{Shaded}

\newpage

\begin{Shaded}
\begin{Highlighting}[]
\CommentTok{# Equipe C}

\NormalTok{## Config 1}
\NormalTok{recpars1 <-}\StringTok{ }\KeywordTok{list}\NormalTok{(}\DataTypeTok{name =} \StringTok{"recombination_blxAlphaBeta"}\NormalTok{, }\DataTypeTok{alpha =} \DecValTok{0}\NormalTok{, }\DataTypeTok{beta =} \DecValTok{0}\NormalTok{)}
\NormalTok{mutpars1 <-}\StringTok{ }\KeywordTok{list}\NormalTok{(}\DataTypeTok{name =} \StringTok{"mutation_rand"}\NormalTok{, }\DataTypeTok{f =} \DecValTok{4}\NormalTok{)}

\NormalTok{## Config 2}
\NormalTok{recpars2 <-}\StringTok{ }\KeywordTok{list}\NormalTok{(}\DataTypeTok{name =} \StringTok{"recombination_linear"}\NormalTok{)}
\NormalTok{mutpars2 <-}\StringTok{ }\KeywordTok{list}\NormalTok{(}\DataTypeTok{name =} \StringTok{"mutation_rand"}\NormalTok{, }\DataTypeTok{f =} \FloatTok{1.5}\NormalTok{)}

\NormalTok{## Config 3}
\NormalTok{recpars3 <-}\StringTok{ }\KeywordTok{list}\NormalTok{(}\DataTypeTok{name =} \StringTok{"recombination_bin"}\NormalTok{, }\DataTypeTok{cr =} \FloatTok{0.25}\NormalTok{) }
\NormalTok{mutpars3 <-}\StringTok{ }\KeywordTok{list}\NormalTok{(}\DataTypeTok{name =} \StringTok{"mutation_best"}\NormalTok{, }\DataTypeTok{f =} \DecValTok{5}\NormalTok{)}

\NormalTok{## Config 4}
\NormalTok{recpars4 <-}\StringTok{ }\KeywordTok{list}\NormalTok{(}\DataTypeTok{name =} \StringTok{"recombination_eigen"}\NormalTok{, }
                 \DataTypeTok{othername =} \StringTok{"recombination_bin"}\NormalTok{, }\DataTypeTok{cr =} \FloatTok{0.5}\NormalTok{) }
\NormalTok{mutpars4 <-}\StringTok{ }\KeywordTok{list}\NormalTok{(}\DataTypeTok{name =} \StringTok{"mutation_best"}\NormalTok{, }\DataTypeTok{f =} \DecValTok{1}\NormalTok{)}
\end{Highlighting}
\end{Shaded}

\begin{Shaded}
\begin{Highlighting}[]
\CommentTok{# Equipe D}
\NormalTok{## Config 1}
\NormalTok{recpars1 <-}\StringTok{ }\KeywordTok{list}\NormalTok{(}\DataTypeTok{name =} \StringTok{"recombination_blxAlphaBeta"}\NormalTok{, }\DataTypeTok{alpha =} \FloatTok{0.4}\NormalTok{, }\DataTypeTok{beta =} \FloatTok{0.4}\NormalTok{) }
\NormalTok{mutpars1 <-}\StringTok{ }\KeywordTok{list}\NormalTok{(}\DataTypeTok{name =} \StringTok{"mutation_rand"}\NormalTok{, }\DataTypeTok{f =} \DecValTok{4}\NormalTok{)}
\NormalTok{popsize1 <-}\StringTok{ }\DecValTok{230}

\NormalTok{## Config 2}
\NormalTok{recpars2 <-}\StringTok{ }\KeywordTok{list}\NormalTok{(}\DataTypeTok{name =} \StringTok{"recombination_eigen"}\NormalTok{, }
                 \DataTypeTok{othername =} \StringTok{"recombination_bin"}\NormalTok{, }\DataTypeTok{cr =} \FloatTok{0.9}\NormalTok{) }
\NormalTok{mutpars2 <-}\StringTok{ }\KeywordTok{list}\NormalTok{(}\DataTypeTok{name =} \StringTok{"mutation_best"}\NormalTok{, }\DataTypeTok{f =} \FloatTok{2.8}\NormalTok{)}
\NormalTok{popsize2 <-}\StringTok{ }\DecValTok{85}

\NormalTok{## Config 3}
\NormalTok{recpars3 <-}\StringTok{ }\KeywordTok{list}\NormalTok{(}\DataTypeTok{name =} \StringTok{"recombination_mmax"}\NormalTok{, }\DataTypeTok{lambda =} \FloatTok{0.25}\NormalTok{)}
\NormalTok{mutpars3 <-}\StringTok{ }\KeywordTok{list}\NormalTok{(}\DataTypeTok{name =} \StringTok{"mutation_best"}\NormalTok{, }\DataTypeTok{f =} \DecValTok{3}\NormalTok{)}
\NormalTok{popsize3 <-}\StringTok{ }\DecValTok{250}

\NormalTok{## Config 4}
\NormalTok{recpars4 <-}\StringTok{ }\KeywordTok{list}\NormalTok{(}\DataTypeTok{name =} \StringTok{"recombination_npoint"}\NormalTok{, }\DataTypeTok{N =} \DecValTok{10}\NormalTok{)}
\NormalTok{mutpars4 <-}\StringTok{ }\KeywordTok{list}\NormalTok{(}\DataTypeTok{name =} \StringTok{"mutation_rand"}\NormalTok{, }\DataTypeTok{f =} \FloatTok{0.8}\NormalTok{)}
\NormalTok{popsize4 <-}\StringTok{ }\DecValTok{200}
\end{Highlighting}
\end{Shaded}

\newpage

\begin{Shaded}
\begin{Highlighting}[]
\CommentTok{# Equipe E}
\NormalTok{## Config 1}
\NormalTok{recpars1 <-}\StringTok{ }\KeywordTok{list}\NormalTok{(}\DataTypeTok{name =} \StringTok{"recombination_lbga"}\NormalTok{)}
\NormalTok{mutpars1 <-}\StringTok{ }\KeywordTok{list}\NormalTok{(}\DataTypeTok{name =} \StringTok{"mutation_rand"}\NormalTok{, }\DataTypeTok{f =} \FloatTok{4.5}\NormalTok{)}

\NormalTok{## Config 2}
\NormalTok{recpars2 <-}\StringTok{ }\KeywordTok{list}\NormalTok{(}\DataTypeTok{name =} \StringTok{"recombination_blxAlphaBeta"}\NormalTok{, }\DataTypeTok{alpha =} \FloatTok{0.1}\NormalTok{, }\DataTypeTok{beta =} \FloatTok{0.1}\NormalTok{)}
\NormalTok{mutpars2 <-}\StringTok{ }\KeywordTok{list}\NormalTok{(}\DataTypeTok{name =} \StringTok{"mutation_rand"}\NormalTok{, }\DataTypeTok{f =} \DecValTok{3}\NormalTok{)}

\NormalTok{## Config 3}
\NormalTok{recpars3 <-}\StringTok{ }\KeywordTok{list}\NormalTok{(}\DataTypeTok{name =} \StringTok{"recombination_blxAlphaBeta"}\NormalTok{, }\DataTypeTok{alpha =} \FloatTok{0.4}\NormalTok{, }\DataTypeTok{beta =} \FloatTok{0.4}\NormalTok{) }
\NormalTok{mutpars3 <-}\StringTok{ }\KeywordTok{list}\NormalTok{(}\DataTypeTok{name =} \StringTok{"mutation_rand"}\NormalTok{, }\DataTypeTok{f =} \DecValTok{4}\NormalTok{)}

\NormalTok{## Config 4}
\NormalTok{recpars4 <-}\StringTok{ }\KeywordTok{list}\NormalTok{(}\DataTypeTok{name =} \StringTok{"recombination_eigen"}\NormalTok{, }
                 \DataTypeTok{othername =} \StringTok{"recombination_bin"}\NormalTok{, }\DataTypeTok{cr =} \FloatTok{0.9}\NormalTok{) }
\NormalTok{mutpars4 <-}\StringTok{ }\KeywordTok{list}\NormalTok{(}\DataTypeTok{name =} \StringTok{"mutation_best"}\NormalTok{, }\DataTypeTok{f =} \FloatTok{2.8}\NormalTok{)}
\end{Highlighting}
\end{Shaded}

\begin{Shaded}
\begin{Highlighting}[]
\CommentTok{# Equipe F}

\NormalTok{## Config 1}
\NormalTok{recpars1 <-}\StringTok{ }\KeywordTok{list}\NormalTok{(}\DataTypeTok{name =} \StringTok{"recombination_mmax"}\NormalTok{, }\DataTypeTok{lambda =} \FloatTok{0.25}\NormalTok{)}
\NormalTok{mutpars1 <-}\StringTok{ }\KeywordTok{list}\NormalTok{(}\DataTypeTok{name =} \StringTok{"mutation_best"}\NormalTok{, }\DataTypeTok{f =} \DecValTok{4}\NormalTok{)}

\NormalTok{## Config 2}
\NormalTok{recpars2 <-}\StringTok{ }\KeywordTok{list}\NormalTok{(}\DataTypeTok{name =} \StringTok{"recombination_npoint"}\NormalTok{, }\DataTypeTok{N =} \DecValTok{17}\NormalTok{)}
\NormalTok{mutpars2 <-}\StringTok{ }\KeywordTok{list}\NormalTok{(}\DataTypeTok{name =} \StringTok{"mutation_rand"}\NormalTok{, }\DataTypeTok{f =} \FloatTok{2.2}\NormalTok{)}

\NormalTok{## Config 3}
\NormalTok{recpars3 <-}\StringTok{ }\KeywordTok{list}\NormalTok{(}\DataTypeTok{name =} \StringTok{"recombination_lbga"}\NormalTok{)}
\NormalTok{mutpars3 <-}\StringTok{ }\KeywordTok{list}\NormalTok{(}\DataTypeTok{name =} \StringTok{"mutation_rand"}\NormalTok{, }\DataTypeTok{f =} \FloatTok{4.5}\NormalTok{)}

\NormalTok{## Config 4}
\NormalTok{recpars4 <-}\StringTok{ }\KeywordTok{list}\NormalTok{(}\DataTypeTok{name =} \StringTok{"recombination_blxAlphaBeta"}\NormalTok{, }\DataTypeTok{alpha =} \FloatTok{0.1}\NormalTok{, }\DataTypeTok{beta =} \FloatTok{0.4}\NormalTok{)}
\NormalTok{mutpars4 <-}\StringTok{ }\KeywordTok{list}\NormalTok{(}\DataTypeTok{name =} \StringTok{"mutation_rand"}\NormalTok{, }\DataTypeTok{f =} \DecValTok{3}\NormalTok{)}
\end{Highlighting}
\end{Shaded}

\newpage

\begin{Shaded}
\begin{Highlighting}[]
\CommentTok{# Equipe G}

\NormalTok{## Config 1}
\NormalTok{recpars1 <-}\StringTok{ }\KeywordTok{list}\NormalTok{(}\DataTypeTok{name =} \StringTok{"recombination_blxAlphaBeta"}\NormalTok{, }\DataTypeTok{alpha =} \DecValTok{0}\NormalTok{, }\DataTypeTok{beta =} \DecValTok{0}\NormalTok{)}
\NormalTok{mutpars1 <-}\StringTok{ }\KeywordTok{list}\NormalTok{(}\DataTypeTok{name =} \StringTok{"mutation_rand"}\NormalTok{, }\DataTypeTok{f =} \DecValTok{4}\NormalTok{)}

\NormalTok{## Config 2}
\NormalTok{recpars2 <-}\StringTok{ }\KeywordTok{list}\NormalTok{(}\DataTypeTok{name =} \StringTok{"recombination_exp"}\NormalTok{, }\DataTypeTok{cr =} \FloatTok{0.6}\NormalTok{)}
\NormalTok{mutpars2 <-}\StringTok{ }\KeywordTok{list}\NormalTok{(}\DataTypeTok{name =} \StringTok{"mutation_best"}\NormalTok{, }\DataTypeTok{f =} \DecValTok{2}\NormalTok{)}

\NormalTok{## Config 3}
\NormalTok{recpars3 <-}\StringTok{ }\KeywordTok{list}\NormalTok{(}\DataTypeTok{name =} \StringTok{"recombination_mmax"}\NormalTok{, }\DataTypeTok{lambda =} \FloatTok{0.25}\NormalTok{)}
\NormalTok{mutpars3 <-}\StringTok{ }\KeywordTok{list}\NormalTok{(}\DataTypeTok{name =} \StringTok{"mutation_best"}\NormalTok{, }\DataTypeTok{f =} \DecValTok{4}\NormalTok{)}

\NormalTok{## Config 4}
\NormalTok{recpars4 <-}\StringTok{ }\KeywordTok{list}\NormalTok{(}\DataTypeTok{name =} \StringTok{"recombination_npoint"}\NormalTok{, }\DataTypeTok{N =} \DecValTok{17}\NormalTok{)}
\NormalTok{mutpars4 <-}\StringTok{ }\KeywordTok{list}\NormalTok{(}\DataTypeTok{name =} \StringTok{"mutation_rand"}\NormalTok{, }\DataTypeTok{f =} \FloatTok{2.2}\NormalTok{)}
\end{Highlighting}
\end{Shaded}

\begin{Shaded}
\begin{Highlighting}[]
\CommentTok{# Equipe H}

\NormalTok{## Config 1}
\NormalTok{recpars1 <-}\StringTok{ }\KeywordTok{list}\NormalTok{(}\DataTypeTok{name =} \StringTok{"recombination_eigen"}\NormalTok{, }
                 \DataTypeTok{othername =} \StringTok{"recombination_bin"}\NormalTok{, }\DataTypeTok{cr =} \FloatTok{0.9}\NormalTok{) }
\NormalTok{mutpars1 <-}\StringTok{ }\KeywordTok{list}\NormalTok{(}\DataTypeTok{name =} \StringTok{"mutation_best"}\NormalTok{, }\DataTypeTok{f =} \FloatTok{2.8}\NormalTok{)}

\NormalTok{## Config 2}
\NormalTok{recpars2 <-}\StringTok{ }\KeywordTok{list}\NormalTok{(}\DataTypeTok{name =} \StringTok{"recombination_sbx"}\NormalTok{, }\DataTypeTok{eta =} \DecValTok{90}\NormalTok{)}
\NormalTok{mutpars2 <-}\StringTok{ }\KeywordTok{list}\NormalTok{(}\DataTypeTok{name =} \StringTok{"mutation_best"}\NormalTok{, }\DataTypeTok{f =} \FloatTok{4.5}\NormalTok{)}

\NormalTok{## Config 3}
\NormalTok{recpars3 <-}\StringTok{ }\KeywordTok{list}\NormalTok{(}\DataTypeTok{name =} \StringTok{"recombination_exp"}\NormalTok{, }\DataTypeTok{cr =} \FloatTok{0.6}\NormalTok{)}
\NormalTok{mutpars3 <-}\StringTok{ }\KeywordTok{list}\NormalTok{(}\DataTypeTok{name =} \StringTok{"mutation_best"}\NormalTok{, }\DataTypeTok{f =} \DecValTok{2}\NormalTok{)}
    
\NormalTok{## Config 4}
\NormalTok{recpars4 <-}\StringTok{ }\KeywordTok{list}\NormalTok{(}\DataTypeTok{name =} \StringTok{"recombination_geo"}\NormalTok{, }\DataTypeTok{alpha =} \FloatTok{0.6}\NormalTok{)}
\NormalTok{mutpars4 <-}\StringTok{ }\KeywordTok{list}\NormalTok{(}\DataTypeTok{name =} \StringTok{"mutation_rand"}\NormalTok{, }\DataTypeTok{f =} \FloatTok{1.2}\NormalTok{)}
\end{Highlighting}
\end{Shaded}

\newpage

\begin{Shaded}
\begin{Highlighting}[]
\CommentTok{# Equipe I}

\NormalTok{## Config 1}
\NormalTok{recpars1 <-}\StringTok{ }\KeywordTok{list}\NormalTok{(}\DataTypeTok{name =} \StringTok{"recombination_blxAlphaBeta"}\NormalTok{, }\DataTypeTok{alpha =} \FloatTok{0.4}\NormalTok{, }\DataTypeTok{beta =} \FloatTok{0.4}\NormalTok{) }
\NormalTok{mutpars1 <-}\StringTok{ }\KeywordTok{list}\NormalTok{(}\DataTypeTok{name =} \StringTok{"mutation_rand"}\NormalTok{, }\DataTypeTok{f =} \DecValTok{4}\NormalTok{)}

\NormalTok{## Config 2}
\NormalTok{recpars2 <-}\StringTok{ }\KeywordTok{list}\NormalTok{(}\DataTypeTok{name =} \StringTok{"recombination_wright"}\NormalTok{)}
\NormalTok{mutpars2 <-}\StringTok{ }\KeywordTok{list}\NormalTok{(}\DataTypeTok{name =} \StringTok{"mutation_best"}\NormalTok{, }\DataTypeTok{f =} \FloatTok{4.8}\NormalTok{)}

\NormalTok{## Config 3}
\NormalTok{recpars3 <-}\StringTok{ }\KeywordTok{list}\NormalTok{(}\DataTypeTok{name =} \StringTok{"recombination_arith"}\NormalTok{) }
\NormalTok{mutpars3 <-}\StringTok{ }\KeywordTok{list}\NormalTok{(}\DataTypeTok{name =} \StringTok{"mutation_rand"}\NormalTok{, }\DataTypeTok{f =} \DecValTok{4}\NormalTok{)}

\NormalTok{## Config 4}
\NormalTok{recpars4 <-}\StringTok{ }\KeywordTok{list}\NormalTok{(}\DataTypeTok{name =} \StringTok{"recombination_bin"}\NormalTok{, }\DataTypeTok{cr =} \FloatTok{0.7}\NormalTok{) }
\NormalTok{mutpars4 <-}\StringTok{ }\KeywordTok{list}\NormalTok{(}\DataTypeTok{name =} \StringTok{"mutation_best"}\NormalTok{, }\DataTypeTok{f =} \DecValTok{3}\NormalTok{)}
\end{Highlighting}
\end{Shaded}

Cada observação individual do desempenho do algoritmo equipado com um
dado operador pode ser obtida através dos comandos abaixo:

\begin{Shaded}
\begin{Highlighting}[]
\KeywordTok{suppressPackageStartupMessages}\NormalTok{(}\KeywordTok{library}\NormalTok{(ExpDE))}

\CommentTok{# Run algorithm on problem:}
\NormalTok{out <-}\StringTok{ }\KeywordTok{ExpDE}\NormalTok{(}\DataTypeTok{mutpars  =}\NormalTok{ mutparsX, }
             \DataTypeTok{recpars  =}\NormalTok{ recparsX, }
             \DataTypeTok{popsize  =}\NormalTok{ popsizeN, }
             \DataTypeTok{selpars  =}\NormalTok{ selpars, }
             \DataTypeTok{stopcrit =}\NormalTok{ stopcritN, }
             \DataTypeTok{probpars =}\NormalTok{ probparsN,}
             \DataTypeTok{showpars =} \KeywordTok{list}\NormalTok{(}\DataTypeTok{show.iters =} \StringTok{"dots"}\NormalTok{, }\DataTypeTok{showevery  =} \DecValTok{20}\NormalTok{))}

\CommentTok{# Extract observation:}
\NormalTok{out}\OperatorTok{$}\NormalTok{Fbest}
\end{Highlighting}
\end{Shaded}

\noindent onde \emph{mutparsX}, \emph{recparsX}, \emph{popsizeN},
\emph{stopcritN} e \emph{probparsN} devem ser substituídos pelas
variáveis apropriadas (\emph{X} é relativo à configuração, e \emph{N} à
dimensão do problema, conforme definições acima).

\subsection{Outras definições}\label{outras-definicoes}

Este estudo de caso consiste das seguintes etapas:

\begin{enumerate}
\def\labelenumi{\arabic{enumi}.}
\tightlist
\item
  Formulação das hipóteses de teste;
\item
  Cálculo do tamanho amostral;
\item
  Coleta e tabulação dos dados;
\item
  Teste das hipóteses;
\item
  Estimação da magnitude da diferença entre os métodos (incluindo
  intervalo de confiança);
\item
  Verificação das premissas dos testes;
\item
  Derivação de conclusões;
\item
  Discussão sobre possíveis limitações do estudo e sugestões de
  melhoria.
\end{enumerate}

Lembre-se que as conclusões devem ser colocadas no contexto das
perguntas técnicas de interesse.

\textbf{Importante:} Estudar atentamente os slides da unidade 11
(\emph{Blocking}) e a seção sobre \emph{Planejamento Completo em Blocos
Aleatorizados} do livro do Montgomery.

\subsection{Relatório}\label{relatorio}

Cada equipe deverá entregar um relatório detalhando o experimento e a
análise dos dados. O relatório será avaliado de acordo com os seguintes
critérios:

\begin{itemize}
\tightlist
\item
  Obediência ao formato determinado (ver abaixo);
\item
  Reproducibilidade dos resultados;
\item
  Qualidade técnica;
\item
  Estrutura da argumentação;
\item
  Correto uso da linguagem (gramática, ortografia, etc.);
\end{itemize}

O relatório deve \emph{obrigatoriamente} ser produzido utilizando
\href{http://rmarkdown.rstudio.com}{R Markdown} (opcionalmente
utilizando estilos distintos, como o do presente documento), e deve
conter todo o código necessário para a reprodução da análise obtida,
embutido na forma de blocos de código no documento. Os grupos devem
enviar:

\begin{itemize}
\tightlist
\item
  O arquivo \textbf{.Rmd} para geração do relatório.
\item
  O arquivo \textbf{.pdf} compilado a partir do \textbf{.Rmd}.
\item
  O arquivo de dados utilizado, em formato \textbf{.csv}.
\end{itemize}

O arquivo \textbf{.Rmd} deve ser capaz de ser compilado em um pdf sem
erros, e deve assumir que o arquivo de dados se encontra no mesmo
diretório do arquivo do relatório. Modelos de estudos de caso estão
disponíveis no repositório da disciplina no github. Caso a equipe deseje
utilizar o estilo do presente documento, pode consultar seu código-fonte
no repositório (note que o mesmo requer a instalação do pacote
\emph{pinp}).

\textbf{Importante}: Salve seu arquivo \textbf{.Rmd} em UTF-8 (para
evitar erros na compilação em outros sistemas). \textbf{Importante}:
Inclua no relatório os papéis desempenhados por cada membro da equipe
(Relator, Verificador etc.)

Relatórios serão aceitos em português ou inglês.

\subsection{Entrega}\label{entrega}

Os arquivos relativos a este estudo de caso (pdf + rmd + csv) deverão
ser comprimidos em um .ZIP e submetidos via Moodle, na atividade
\textbf{Case Study 05}, até a data-limite de \textbf{Sexta-feira, 6 de
julho de 2018, às 23:55h}. Após esta data o sistema estará fechado para
recebimento.

\textbf{Importante}: Apenas uma submissão por equipe é necessária.

\textbf{Importante}: Relatórios não serão recebidos por e-mail ou em
formato impresso.

%\showmatmethods


\bibliography{CS02.bib}
\bibliographystyle{jss}



\end{document}

